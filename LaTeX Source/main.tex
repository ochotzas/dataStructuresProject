\title{Εργασία στις Δομές Δεδομένων 2021}
\documentclass[a4paper]{article}

\usepackage[english,greek]{babel}
\usepackage[LGR]{fontenc}
\usepackage[utf8]{inputenc}
\usepackage[greek]{babel}
\usepackage{lmodern}
\usepackage{amsmath}

\usepackage{ucs}
\newcommand{\en}{\selectlanguage{english}}
\newcommand{\gr}{\selectlanguage{greek}}

%\usepackage{XCharter}
\usepackage{alphabeta}
\usepackage{abstract}
\usepackage{fancyhdr}
\usepackage{minted}
\usepackage{xcolor}

% Γεωμετρικότητα των σελιδών σε mm
\usepackage{geometry}
 \geometry{
 a4paper,
 %total={140mm,257mm},
 left=15mm,
 right=15mm,
 top=20mm,
 bottom=20mm
 }

\renewcommand{\listingscaption}{Πρόγραμμα}
\renewcommand{\abstractname}{Περίληψη}
\renewcommand{\headrulewidth}{0.5pt}

% Επεξεργασία αρίθμησης σχημάτων & πινάκων
\counterwithin{listing}{section}
\counterwithin{table}{section}

\usemintedstyle{borland}

\definecolor{LightGray}{gray}{0.9}
\definecolor{DarkGray}{gray}{0.1}
\definecolor{codegreen}{rgb}{0,0.6,0}
\definecolor{codegray}{rgb}{0.5,0.5,0.5}
\definecolor{codepurple}{rgb}{0.58,0,0.82}
\definecolor{backcolour}{rgb}{0.95,0.95,0.92}

% Στυλ σελίδας
\pagestyle{fancy}
\renewcommand{\sectionmark}[1]{\markright{#1}}
\renewcommand{\subsectionmark}[1]{}

\fancyhf{}
\fancyhead[L]{\rightmark}
\fancyhead[R]{Εργασία στις Δομές Δεδομένων}
\rfoot{Σελίδα \thepage}

% Πληροφορίες και σχετικές βιβλ. εγγράφου
\begin{document}

\title{Εργασία στις Δομές Δεδομένων}
\author{Νικόλαος Βογιατζής (3952), Ολγκέρ Χότζα (3899)}
\date{Εαρινό εξάμηνο 2021}

\maketitle

\begin{abstract}
    \begin{center}
        Στις παρούσες διαφάνειες θα παρουσιαστούν μερικές δομές δεδομένων ως προς την λειτουργία και την υλοποίηση τους σε γλώσσα προγραμματισμού \selectlanguage{english}C++. H κάθε δομή αξιολογείται με βάση τον χρόνο εκτέλεσης και τερματισμού κάποιον συγκεκριμένων ενεργειών/αλγορίθμων.
    \end{center}
\end{abstract}

\tableofcontents

% Αρχή της εργασίας
\section{Εισαγωγή}
Οι δομές που έχουμε υλοποιήσει και αξιολογήσει είναι οι παρακάτω πέντε:
\begin{enumerate}
    \item αταξινόμητος πίνακας,
    \item ταξινομημένος πίνακας,
    \item απλό δυαδικό δέντρο αναζήτησης,
    \item δυαδικό δένδρο αναζήτησης τύπου \selectlanguage{english}AVL και
    \item πίνακας κατακερματισμού με ανοικτή διεύθυνση.
\end{enumerate}

Για τις προαναφερθείσες δομές δεδομένων, με βάση το κείμενο από το \selectlanguage{english}Gutenberg Project (https://www.gutenberg.org/) που περιέχει γνωστά έργα παγκόσμιας λογοτεχνίας, έχουμε καταγράψει τους χρόνους ως προς την εισαγωγή και αναζήτηση.

\gr
\section{Αταξινόμητος πίνακας}
\subsection*{Περιγραφή}

Η δομή ενός μη ταξινομημένου πίνακα είναι μια συλλογή στοιχείων όπου κάθε στοιχείο κατέχει μια σχετική θέση σε σχέση με τα άλλα.

Γενικά, ένας μη ταξινομημένος πίνακας είναι μια δομή δεδομένων που τα στοιχεία του είναι διατεταγμένα με βάση τη σειρά που εισήχθησαν στον πίνακα. Κάθε στοιχείο του πίνακα διαπερνάται με τη σειρά, έτσι για παράδειγμα η αναζήτηση γίνεται σειριακά έως ότου βρεθεί το επιθυμητό στοιχείο, διαφορετικά η σύγκριση συνεχίζεται μέχρι τέλους του πίνακα.


\begin{table}[h!]
\centering
\begin{tabular}{||c c c c||} 
 \hline
 Θέση $i$ στοιχείου & Περίπτωση & Συγκρίσεις & Πολυπλοκότητα \\
 \hline\hline
 $0$ & Καλύτερή & $1$ & $O(1)$ \\ 
 $n-1$ & Χειρότερή & $n$ & $O(n)$ \\ 
 \hline
\end{tabular}
\caption{Περιπτώσεις της σειριακής αναζήτησης}
\label{table:1}
\end{table}


Για να μελετήσουμε θεωρητικά τις δυο περιπτώσεις ενός αλγορίθμου θα πρέπει να το δούμε υπό μαθηματικής άποψης. Έστω ότι $n$ το σύνολο των στοιχείων και $i$ η θέση του κάθε στοιχείου στον πίνακα. Η \textbf{καλύτερη περίπτωση} είναι το στοιχείο που αναζητούμε να βρίσκεται στη πρώτη θέση του πίνακα ($i=0$). Αντίστοιχα, η \textbf{χειρότερη περίπτωση} θα ήταν αν το στοιχείο που αναζητούμε είναι το τελευταίο στοιχείο του πίνακα ($i=n-1$) ή το στοιχείο που αναζητούμε να μην περιεχόταν στο πίνακα (βλ. Πίνακας \ref{table:1}).

Πριν ξεκινήσουμε να υλοποιούμε την εισαγωγή, αναζήτηση και διαγραφή του μη ταξινομημένου πίνακα \en (unorderedArray), \gr  θα πρέπει να ξεκινήσουμε την υλοποίηση της κλάσης.


\en
\begin{listing}[ht]
\begin{minted}[frame=lines, framesep=1em]{cpp}

#pragma once
#include <string>

using std::string;

class unorderedArray
{
private:
    struct Arr
    {
        string value;
        long long unsigned int times;
    };
    Arr* Array;
    long long unsigned int FirstFreeCell;
protected:
    bool ReallocateArray(int);
public:
    unorderedArray();
    ~unorderedArray();
    
    void Insert(string);  
    bool Delete(string);
    bool Find(string, long long unsigned int&, long long unsigned int&);
    long long unsigned int getFirstFreeCell() { return FirstFreeCell; }

    string& operator[](unsigned int possition) { return Array[possition].value; }
};

\end{minted}
\caption{Υλοποίηση της κλάσης για τον αταξινόμητο πίνακα}
\label{listing:1}
\end{listing}
\gr


Όπως φαίνεται παραπάνω (Πρόγραμμα \ref{listing:1}), κατασκευάζουμε εντός της κλάσης της \en UnorderedArray \gr ένα \en struct \gr το όποιο αποθηκεύει την λέξη μας καθώς και πόσες φορές αποθηκεύεται στον παρακάτω πίνακα.

Διατηρούμε την πρώτη ελεύθερη θέση σε μια μεταβλητή τύπου \en long long unsigned int \gr για να βεβαιωθούμε πως θα φτάσουν τα \en bytes \gr για την αποθήκευση του πλήθους λέξεων που θα αποθηκευτούν $+1$ θέση, όπου θα είναι η ελεύθερη θέση.

Στο δημόσιο τμήμα της κλάσης της αταξινόμητης τομής, υπάρχει ο κατασκευαστής, καταστροφέας και διάφορες συναρτήσεις τις οποίες θα μελετήσουμε παρακάτω.

\subsection{Μέθοδος εισαγωγής}

Ας υποθέσουμε ότι έχουμε έναν δυναμικό πίνακα με $n$ θέσεις και $i$ θα ονομάσουμε το κάθε στοιχείο του πίνακα με $i\in[0,n)$ και $j$ το στοιχείο που θέλουμε να εισάγουμε στον πίνακα. Σε έναν μη ταξινομημένο πίνακα, η λειτουργία εισαγωγής είναι ταχύτερη σε σύγκριση με έναν ταξινομημένο πίνακα επειδή δεν χρειάζεται να ενδιαφερόμαστε για τη θέση στην οποία πρόκειται να τοποθετηθεί το στοιχείο.

Κάθε στοιχείο $j$ θα τοποθετείτε στην τελευταία θέση του πίνακα. Για να λύσουμε το θέμα του πλήρη πίνακα, κάθε φορά θα δεσμεύουμε $n+1$ θέσεις για κάθε $j$ στοιχείο. Επομένως, ας δημιουργήσουμε δύο συναρτήσεις της κλάσης μας όπου η πρώτη συνάρτηση θα πραγματοποιεί την εισαγωγή στοιχείων στην τελευταία θέση του πίνακα και δεύτερων τη συνάρτηση που θα δεσμεύει νέα μνήμη.

Πρακτικά, στην δική μας υλοποίηση, λαμβάνοντας υπόψιν ότι θέλουμε να γνωρίζουμε και το πλήθος εμφανίσεων κάθε λέξης, χρησιμοποιούμε την συνάρτηση \en Find \gr για να εντοπίσουμε αν η λέξη που θα εισάγουμε υπάρχει ή οχι. Σε περίπτωση που υπάρχει, τότε απλά αυξάνουμε κατά μια μονάδα την μεταβλητή εμφάνισης της λέξης. Σε διαφορετική περίπτωση, τότε δεσμεύουμε μια μονάδα περισσότερο χώρο (βλ. Πρόγραμμα \ref{listing:2} - Συνάρτηση \en ReallocateArray(int) \gr ) για την \en Array\gr. Στη συνέχεια, αφού έχει δεσμευτεί χώρος, απλά εισάγουμε την λέξη στον πίνακα και θέτουμε ότι η λέξη είναι η πρώτη (δηλαδή έχει πλήθος εμφανίσεων μια φορές).


\en
\begin{listing}[ht]
\begin{minted}[frame=lines, framesep=1em]{cpp}

void unorderedArray::Insert(string word)
{
    long long unsigned int temp, times;
    if(Find(word, temp, times))
        Array[temp].times++;
    else
    {
        ReallocateArray(1);
        Array[First_Free_Element-1].value = word;
        Array[First_Free_Element-1].times = 1;
    }
}

bool unorderedArray::ReallocateArray(int n)
{
    Arr* temp = new (nothrow) Arr[First_Free_Element+n];

    if (temp == nullptr) return false;

    for (long long unsigned int element = 0; element < First_Free_Element; element++)
    {
        temp[element].value = Array[element].value;
        temp[element].times= Array[element].times;
    }

    delete[] Array;
    Array = temp;

    First_Free_Element++;
    return true;
}

\end{minted}
\caption{Δημιουργία των συναρτήσεων Insert και RelocateArray του αταξινόμητου πίνακα}
\label{listing:2}
\end{listing}
\gr


\subsection{Μέθοδος αναζήτησής}

Σε έναν μη ταξινομημένο πίνακα, η λειτουργία αναζήτησης θα εκτελεστεί με γραμμική διέλευση από το πρώτο στοιχείο ($i=0$) στο τελευταίο στοιχείο ($i=n-1$). Η διαδικασία αυτή ως προς την ταχύτητα εύρεσης ενός στοιχείου $k$ (όπου $k$ το ενδιαφερόμενο στοιχείο), παρουσιάζει καθυστέρησή ανάλογα με τη θέση $i$ που βρίσκεται το στοιχείο $k$.

Με βάση τον Πίνακα \ref{table:1} παρατηρούμε ότι υπάρχει χειρότερη περίπτωση $n$ συγκρίσεων σε περίπτωση που το $k$ στοιχείο βρίσκεται στη θέση $n-1$ του πίνακα ή δεν περιέχεται. Έτσι λοιπόν, ο αλγόριθμος της σειριακής αναζήτησης έχει πολυπλοκότητα χειρότερης περίπτωσής $O(n)$.

Όπως φαίνεται στο Πρόγραμμα \ref{listing:3}, κάθε φορά που καλείται η συνάρτηση, γίνεται έλεγχος σε περίπτωση που ο πίνακας είναι άδειος. Για κάθε περίπτωση που υπάρχουν έστω ένα ή περισσότερα στοιχεία/λέξεις μέσα στον πίνακα, τότε σειριακά επιτελείται η αναζήτηση. Αν το επιθυμητό στοιχείο βρεθεί, τότε, με αναφορά επιστρέφεται η θέση στην οποία βρέθηκε το στοιχείο, πόσες φορές βρέθηκε (που υπάρχει ήδη από την εισαγωγή) και τέλος επιστρέφεται ότι βρέθηκε. Για κάθε άλλη περίπτωση επιστρέφονται -- με αναφορά -- μηδενικές εμφανίσεις και \en false\gr.

\en
\begin{listing}[ht]
\begin{minted}[frame=lines, framesep=1em]{cpp}

bool unorderedArray::Find(string object, int &temp, int &existance_times)
{
    if(First_Free_Element == 0)
        return false;
        
    for (int i = 0; i < First_Free_Element; i++)
    {
        if(Array[i].value==object)
        {
            temp = i;
            existance_times = Array[i].times;
            return true;
        }
    }

    existance_times = 0;
    return false;
}

\end{minted}
\caption{Δημιουργία της συνάρτησής Find του αταξινόμητου πίνακα}
\label{listing:3}
\end{listing}

\subsection{Μέθοδος διαγραφής}

Η διαγραφή στον αταξινόμητο πίνακα είναι χρονοβόρα καθώς γίνεται χρήση της συνάρτησης \en Find \gr και διαπερνούμε όλα τα στοιχεία του πίνακα. Η μεθοδολογία που ακολουθούμε για να καλύψουμε τις κενές θέσεις είναι να αλλάζουμε την θέση του κάθε στοιχείου του πίνακα κατά μια θέση αρνητικά, με βάση τη θέση την οποία βρέθηκε το επιθυμητό στοιχείο που είναι προς διαγραφή.

Υπό αλγοριθμικής σκέψης, θα λέγαμε ότι, αν $k$ το στοιχείο προς αναζήτηση και $P(k)$ η θέση στην οποία βρέθηκε, τότε όλα τα στοιχεία με $ \mbox{θέσεις} \geq P(k)$ θα αντιμετατεθούν τις θέσεις τους κατά μείον μια θέση από την αρχική τους.

Υλοποιώντας την σκέψη μας με $C++$ κώδικα (βλ. Πρόγραμμα \ref{listing:4}), αρχικά θα ελέγξουμε αν το στοιχείο που επιθυμούμε να διαγράψουμε υπάρχει. Αν δεν υπάρχει τότε τερματίζουμε την συνάρτηση και επιστρέφουμε \en false\gr. Διαφορετικά, υλοποιούμε την παραπάνω σκέψη μας σε αλγόριθμο. Τέλος, μειώνουμε την τελευταία ελεύθερη θέση του πίνακα κατά μια μονάδα και επιστρέφουμε \en true\gr.

\en
\begin{listing}[ht]
\begin{minted}[frame=lines, framesep=1em]{cpp}

bool unorderedArray::Delete(string object)
{
    long long unsigned int temp, existance_times;
    if (!Find(object, temp, existance_times))
        return false;

    for (long long unsigned int position = temp; position < First_Free_Element - 1; position++)
    {
        Array[position].value = Array[position + 1].value;
        Array[position].times = Array[position + 1].times;
    }

    First_Free_Element--;
    return true;
}

\end{minted}
\caption{Δημιουργία της συνάρτησής Delete του αταξινόμητου πίνακα}
\label{listing:4}
\end{listing}

\subsection{Σχολιασμός και χρόνοι εκτέλεσης}

Ο χρόνος εισαγωγής σε έναν αταξινόμητο πίνακα, σαφώς και είναι λιγότερος σε σχέση με την δομή του ταξινομημένου πίνακα. Κάθε στοιχείο μεταφορτώνεται στο τέλος του πίνακα, οπότε, κατ-ανάγκη, δεν την καθιστά, ως προς την εισαγωγή, χειρότερη δομή.

\gr
\begin{table}[!ht]
\centering
\begin{tabular}{||c c c||} 
 \hline
 Δομή  & Χρόνος Εισαγωγής & Χρόνος Αναζήτησης/10χιλ. λέξεις \\
 \hline\hline
 Αταξινόμητος πίνακας & $\approx$10.9451192 ώρες & $\approx$33.1032167 λεπτά \\
 \hline
\end{tabular}
\caption{Χρόνοι εκτέλεσης εισαγωγής και αναζήτησης για τον αταξινόμητο πίνακα}
\label{table:3}
\end{table}

Όσο για την αναζήτηση, επειδή κάθε στοιχείο αναζητείται σειριακά, δηλαδή συγκρίνουμε όλα τα προηγούμενα στοιχεία με το στοιχείο που ψάχνουμε, καθιστά την δομή αργή. Αν δώσουμε μια εκτίμηση για δέκα χιλιάδες λέξεις έχουμε τα παραπάνω χρονικά αποτελέσματα (Πίνακας \ref{table:3}). \clearpage
\section{Ταξινομημένος πίνακας}
\subsection*{Περιγραφή}

Μια δομή ενός ταξινομημένου πίνακα χαρακτηρίζεται από την διάταξη των στοιχείων που ταξινομούνται με κάποια σειρά, όπως αριθμητικό, αλφαβητικό κ.λπ. σε αύξουσα ή φθίνουσα διάταξη. Υπάρχουν πολλοί αλγόριθμοι για την ταξινόμηση ενός λεξικογραφικού πίνακα. Η δομή μας είναι υλοποιημένη με βάση τον αλγόριθμο \en Insertion Sort\gr.

\en
\begin{listing}[ht]
\begin{minted}[frame=lines, framesep=1em]{cpp}

#pragma once
#include <string>

using std::string;

class orderedArray
{
private:
    struct Arr
    {
        string value;
        long long unsigned int times;
    };
    Arr* Array;
    long long unsigned int FirstFreeCell;
protected:
    bool binarySearch(string, long long  int,long long  int, long long int&, long long int&);
    bool ReallocateArray(int);
    void InsertionSort();
public:
    orderedArray();
    ~orderedArray();

    void Insert(string);
    bool Find(string, long long&);
    bool Delete(string);

    long long unsigned int getFirstFreeCell() { return FirstFreeCell; }

    string& operator[](unsigned int possition) { return Array[possition].value; }
};
\end{minted}
\caption{Υλοποίηση της κλάσης για τον ταξινομημένο πίνακα}
\label{listing:5}
\end{listing}

Ξεκινώντας την υλοποίηση της κλάσης, θα πρέπει να διαπραγματευτούμε τις λειτουργίες της μέσα από το \en header \gr αρχείο της (βλ. Πρόγραμμα \ref{listing:5}). Η κλάση του ταξινομημένου πίνακα φαίνεται να έχει ελάχιστες διαφορές ως προς τις μεταβλητές και τις συναρτήσεις. Παρακάτω θα αναλυθούν μόνο οι κατά περίπτωση διαφορετικές συναρτήσεις. Οι κοινές υλοποιήσεις -- όπως είναι η μέθοδος διαγραφής -- φαίνονται στην προηγούμενη δομή.

\subsection{Μέθοδος εισαγωγής}

Όπως φαίνεται στο παρακάτω τμήμα κώδικα (Πρόγραμμα \ref{listing:6}) η συνάρτηση \en Insert \gr δεν διαφέρει πολύ ως προς την υλοποίηση σε σχέση με την  \en Insert \gr του αταξινόμητου πίνακα.

Αρχικά, ελέγχουμε μέσω της \en Binary Search \gr αν υπάρχει η προς εισαγωγή λέξη ήδη μέσα στον ταξινομημένο πίνακα. Σε περίπτωση που η λέξη δεν υπάρχει καθόλου στον πίνακα, τότε εκτελείται το τμήμα κώδικα που βρίσκεται εντός της \en else\gr. Το υπόλοιπο τμήμα του κώδικα για τη συνάρτηση \en Insert \gr είναι κοινό με του αταξινόμητου πίνακα, με τη διαφορά πως μετά την εκχώρηση των τιμών στις μεταβλητές, εκτελείται η συνάρτηση ταξινόμησης του πίνακα (η \en Insertion Sort \gr που θα αναλύσουμε παρακάτω).

\en
\begin{listing}[ht]
\begin{minted}[frame=lines, framesep=1em]{cpp}

void orderedArray::Insert(string word)
{
    long long  int temp ,existance_times;
    if (binary_search(word, 0, LastElement - 1, temp, existance_times))
        Array[temp].times++;
    else
    {
        ReallocateArray(1);
        Array[LastElement - 1].value = word;
        Array[LastElement - 1].times = 1;
        insertionSort();
    }
}

\end{minted}
\caption{Δημιουργία της συνάρτησής Insert του ταξινομημένου πίνακα}
\label{listing:6}
\end{listing}
\gr


\subsection{Μέθοδος αναζήτησής}

Με τη βοήθεια του ταξινομημένου πίνακα, για την αναζήτηση των στοιχείων μπορούμε πλέον να χρησιμοποιήσουμε έναν γρηγορότερο αλγόριθμο για αναζήτηση λέξεων σε σχέση με τον αλγόριθμο σειριακής αναζήτησης.

Η διαδικασία η οποία ακολουθεί ο αλγόριθμος δυαδικής αναζήτησης (\en Binary Search algorithm\gr) σε έναν ταξινομημένο πίνακα είναι:

\begin{enumerate}
    \item Ξεκινάμε με ένα διάστημα που καλύπτει ολόκληρο τον πίνακα.
    \item Διαιρούμε επανειλημμένα το διάστημα αναζήτησης στο μισό.
    \item Εάν η τιμή του κλειδιού αναζήτησης είναι μικρότερη από το στοιχείο στο μέσο του διαστήματος, περιορίζουμε το διάστημα στο κάτω μισό. Διαφορετικά, το περιορίζουμε στο πάνω μισό.
   \item Ελέγχουμε επανειλημμένα έως ότου βρεθεί η τιμή ή το διάστημα είναι κενό.
\end{enumerate}

\textit{(Ο αλγόριθμος της δυαδικής αναζήτησης φαίνεται στο Πρόγραμμα \ref{listing:7})}

\en
\begin{listing}[ht]
\begin{minted}[frame=lines, framesep=1em]{cpp}

bool orderedArray::binary_search(string word, int left, right, int& temp, int& existance_times)
{
    if (right >= left)
    {
        long long  int mid = left + (right - left) / 2;

        if (Array[mid].value == word)
        {
            temp = mid;
            existance_times = Array[mid].times;
            return true;
        }
        
        if (Array[mid].value > word)
            return binary_search(word, left, mid - 1, temp, existance_times);

        return binary_search(word, mid + 1, right, temp , existance_times);
    }
    
    existance_times = 0;
    return false;
}

\end{minted}
\caption{Δημιουργία της συνάρτησής Binary Search του ταξινομημένου πίνακα}
\label{listing:7}
\end{listing}
\gr


\subsection{Μέθοδος ταξινόμησής}

Ο αλγόριθμος \en Insertion Sort \gr είναι ένας απλός αλγόριθμος ταξινόμησης που λειτουργεί παρόμοιος με τον τρόπο ταξινόμησης των παιγνιοχάρτων. Ο πίνακας χωρίζεται ουσιαστικά σε ένα ταξινομημένο και ένα μη ταξινομημένο τμήμα. Οι τιμές από το μη ταξινομημένο τμήμα επιλέγονται και τοποθετούνται στη σωστή θέση στο ταξινομημένο μέρος. 

Σε κάθε επανάληψη, η \en Insertion Sort \gr αφαιρεί ένα στοιχείο από τα δεδομένα εισαγωγής, βρίσκει τη θέση στην οποία ανήκει στη ταξινομημένη λίστα και το εισάγει εκεί. Η διαδικασία αυτή επαναλαμβάνεται έως ότου δεν υπάρχουν άλλα στοιχεία εισόδου (Βλ. αλγόριθμο σε \en C++ \gr στο Πρόγραμμα \ref{listing:8}).

\begin{table}[!h]
\centering
\begin{tabular}{||c c c||} 
 \hline
 Περίπτωση & Συγκρίσεις & Αντιμεταθέσεις \\
 \hline\hline
 Καλύτερή &  $O(n)$ & $O(1)$ \\
 Χειρότερή & $O(n^2)$ & $O(n^2)$ \\ 
 \hline
\end{tabular}
\caption{Πολυπλοκότητα του \en Selection Sort \gr}
\label{table:2}
\end{table}

Για κάθε θέση του πίνακα, ελέγχει την τιμή έναντι της μεγαλύτερης τιμής στη λίστα ταξινόμησης (η οποία συμβαίνει να είναι δίπλα της, δηλαδή στην προηγούμενη θέση του πίνακα που έχει ελεγχθεί). Εάν είναι μεγαλύτερο, αφήνει το στοιχείο στη θέση του και μετακινείται στο επόμενο. Εάν είναι μικρότερο, βρίσκει τη σωστή θέση μέσα στη λίστα ταξινόμησης, μετατοπίζει όλες τις μεγαλύτερες τιμές για να δημιουργήσει κενό και το εισάγει στη σωστή/κενή θέση.

\en
\begin{listing}[ht]
\begin{minted}[frame=lines, framesep=1em]{cpp}

void orderedArray::insertionSort()
{
    int i,j,Time;
    string key;
    for(i = 1; i < LastElement; i++)
    {
        key = Array[i].value;
        Time = Array[i].times;
        j = i - 1;
        while(j >= 0 && Array[j].value > key)
        {
            Array[j + 1].value = Array[j].value;
            Array[j + 1].times = Array[j].times;
            j = j - 1;
        }
        Array[j + 1].value = key;
        Array[j + 1].times = Time;
    }
}

\end{minted}
\caption{Δημιουργία της συνάρτησής Insertion Sort του ταξινομημένου πίνακα}
\label{listing:8}
\end{listing}
\gr

Ο πίνακας που προκύπτει μετά από $i$ επαναλήψεις έχει την ιδιότητα όπου ταξινομούνται οι πρώτες καταχωρήσεις $i+1$ ($+1$ επειδή η πρώτη καταχώρηση παραλείπεται). Σε κάθε επανάληψη η πρώτη εναπομένουσα καταχώριση της εισόδου αφαιρείται και εισάγεται στο αποτέλεσμα, στη σωστή θέση.


\subsection{Σχολιασμός και χρόνοι εκτέλεσης}

Ο χρόνος εισαγωγής σε έναν ταξινομημένο πίνακα, είναι αρκετά περισσότερος σε σχέση με την δομή του αταξινόμητου πίνακα που απλά τα στοιχεία εισάγονται στο τέλος. Λόγω της \en Binary Search \gr που εκτελείται για να ελέγξει αν υπάρχει ήδη το στοιχείο μέσα στον πίνακα και της \en Insertion Sort \gr που όπως είδαμε στον Πίνακα \ref{table:2} έχει χειρότερη περίπτωση $O(n^2)$, όπου μπορούμε να συμπεράνουμε πως αυτό καθιστά αργή την δομή μας ως προς την εισαγωγή.

\gr
\begin{table}[!h]
\centering
\begin{tabular}{||c c c||} 
 \hline
 Δομή & Χρόνος Εισαγωγής & Χρόνος Αναζήτησης/10χιλ. λέξεις \\
 \hline\hline
 Ταξινομημένος πίνακας & $\approx$14.3950753 ώρες & $\approx$1.58043333 λεπτά \\
 \hline
\end{tabular}
\caption{Χρόνοι εκτέλεσης εισαγωγής και αναζήτησης για τον ταξινομημένο πίνακα}
\label{table:4}
\end{table}

Όσο για την αναζήτηση, όπου η διαδικασία που ακολουθείται συμβαίνει αναδρομικά, γνωρίζουμε ότι ο αλγόριθμος της δυαδικής αναζήτησης έχει πολυπλοκότητα $O(log(n))$ που μπορούμε να το διαπιστώσουμε αν αναλύσουμε τον κώδικα του. Αν εξετάσουμε συνολικά την δομή του ταξινομημένου πίνακα για δέκα χιλιάδες λέξεις, έχουμε τα παραπάνω αποτελέσματα (Πίνακας \ref{table:4}). \clearpage
\section{Απλό και δυαδικό δένδρο αναζήτησής τύπου \en AVL \gr}
\subsection*{Περιγραφή}

Ένα απλό δυαδικό δέντρο όπως και ένα δένδρο τύπου \en AVL \gr  είναι μία δενδρική δομή δεδομένων η οποία αποτελείται από κόμβους που, όπως και στα δυαδικά δέντρα, έχουν το πολύ δύο παιδιά. Αυτό που καθιστά το δυαδικό δέντρο αναζήτησης διαφορετικό από τα απλά δυαδικά δέντρα είναι το γεγονός ότι οι κόμβοι έχουν πάντα ως αριστερό παιδί αυτό με μικρότερη τιμή σε σχέση με τη ρίζα και δεξί παιδί αυτό με μεγαλύτερη τιμή σε σχέση με τη ρίζα. Το παραπάνω ισχύει -- με αναδρομική διαδικασία -- για όλα τα υποδέντρα που περιέχονται στο βασικό δυαδικό δέντρο αναζήτησης. Η διαφορά ανάμεσα σε απλό και δένδρο τύπου \en AVL \gr είναι στα ύψη των δύο θυγατρικών υποδέντρων οποιουδήποτε κόμβου, που διαφέρουν το πολύ κατά ένα.

Όπως προαναφέρθηκε τα δέντρα αποτελούνται από κόμβους, οπότε πριν προχωρήσουμε στις λειτουργίες τις εισαγωγής/αναζήτησής/διαγραφής ας δούμε τι περιέχει το \en header \gr  αρχείο (\en node.h \gr) που αφορά τους κόμβους που αποθηκεύουμε στο δέντρο (βλ. Πρόγραμμα \ref{listing:13}).

Αρχικά χρειαζόμαστε ένα \en struct \gr ώστε να αποθηκεύουμε για κάθε κόμβο τη λέξη που επιθυμούμε και τις φορές εμφάνισης της μέσα στη δομή. Έπειτα ως μέλη της κλάσης χρειαζόμαστε τρεις μεταβλητές τύπου \en node \gr που αφορούν το δεξί  και το αριστερό παιδί κάποιας ρίζας και μια \en parent \gr που αφορά τον κόμβο-ρίζα που είναι ο γονέας ενός κόμβου για κάποιο υποδέντρο. Τέλος υπάρχουν δύο κατασκευαστές: Ένας κενός που δημιουργεί κενό κόμβο και ένας δεύτερος με όρισμα ένα \en string \gr που χρησιμοποιείται στην εισαγωγή ενός νέου κόμβου. 

Όλα τα παραπάνω είναι δηλωμένα στο δημόσιο τμήμα της κλάσης παρ’ όλο που δεν χρησιμοποιούνται στη \en main \gr. Αυτό συμβαίνει για απλούστευση του κώδικα.

\en
\begin{listing}[ht]
\begin{minted}[frame=lines, framesep=1em]{cpp}

#pragma once
#include <string>

using std::string;

struct node_info
{
    string value;
    long long int times;
};

class node
{
public:
    node_info data;
    long long int height;
    node* left;
    node* right;
    node* parent;
    node(const string&);
    node();
};

\end{minted}
\caption{Υλοποίηση του struct info και της κλάσης node για τις δενδρικές δομές}
\label{listing:13}
\end{listing}
\gr


\subsection{Μέθοδος εισαγωγής}

Η εισαγωγή σε ένα απλό δυαδικό δέντρο αναζήτησης και ενός δένδρου τύπου \en AVL \gr αποτελεί μια λειτουργία/μέθοδο με πολλαπλούς ελέγχους που διευκολύνουν στη συνέχεια τη μέθοδο αναζήτησης και διαγραφής.

Στον κώδικα της μεθόδου εισαγωγής, αρχικά ελέγχεται αν το δέντρο είναι κενό οπότε και δημιουργείται η ρίζα του δέντρου που περιέχει τη λέξη \en word \gr και η μεταβλητή με τις φορές εμφάνισης τίθεται σε ένα. Για κάθε επόμενο κόμβο που θέλουμε να δημιουργηθεί, ελέγχουμε αν η λέξη (η οποία είναι το κλειδί για τους ελέγχους στην εισαγωγή), είναι μικρότερη ή μεγαλύτερη από τη ρίζα του εκάστοτε δέντρου ή υποδέντρου. Οι έλεγχοι αυτοί επαναλαμβάνονται μέχρι τη στιγμή που θα βρεθεί ένας κόμβος κατάλληλος για την δημιουργία του νέου κόμβου ως παιδί του. Αυτό προϋποθέτει να έχει ο κόμβος-γονέας αριστερό ή δεξί παιδί (ανάλογα τη θέση που πρέπει να εισαχθεί ο νέος κόμβος), ίσο με \en nullptr \gr. Σε αυτήν την περίπτωση έχουμε επιτυχημένη εισαγωγή και η συνάρτηση \en insert \gr τερματίζει επιστρέφοντας τη λογική τιμή \en true \gr. Αξιοσημείωτη είναι η περίπτωση που υπάρχει ήδη ένας κόμβος με αποθηκευμένη λέξη τη \en word \gr που θέλουμε να εισάγουμε. Τη περίπτωση αυτή τη διαχειριζόμαστε αυξάνοντας μόνο τον μετρητή που δείχνει τις φορές εμφάνισης της λέξης κατά μια μονάδα. Έτσι βεβαιωνόμαστε ότι κάθε κόμβος του δέντρου αποθηκεύει μόνο τις λέξεις που είναι άγνωστες για τη δομή. Η διαφορά μεταξύ των δυο δένδρων είναι ότι στο τέλος κάθε εισαγωγής, το δένδρο τύπου \en AVL \gr καλεί τη συνάρτησή εξισορρόπησής αναδρομικά από το κόμβο που εισήχθη έως τη ρίζα του δένδρου, ακολουθώντας το μονοπάτι που ακολουθήθηκε στην \en insert \gr για την εισαγωγή του.  

\subsection{Μέθοδος αναζήτησής}

Η διαδικασία της αναζήτησης σε ένα απλό δυαδικό δέντρο αναζήτησης όπως και σε ένα δένδρο τύπου \en AVL \gr δε διαφέρει και πολύ σε σχέση με την δυαδική αναζήτηση σε έναν ταξινομημένο πίνακα.

Ο κόμβος αναζητείται με βάση την λέξη-κλειδί που έχει αποθηκευμένη. Έτσι ξεκινώντας από τη ρίζα του δέντρου, μέσα σε μία επανάληψη ελέγχουμε αν η λέξη-κλειδί για την αναζήτηση είναι μεγαλύτερη ή μικρότερη της λέξης που έχει αποθηκευμένη η ρίζα (αρχικά στον πρώτο έλεγχο), ώστε στην επόμενη επανάληψη να λάβουμε ως κόμβο προς έλεγχο το αριστερό (ή το δεξί) παιδί της ρίζας, αγνοώντας το δεξί (ή το αριστερό υποδέντρο) αντίστοιχα. Η παραπάνω διαδικασία επαναλαμβάνεται μέχρι να βρεθεί ο κόμβος που περιέχει τη λέξη \en word \gr ή να τελειώσουν οι κόμβοι προς έλεγχο, δηλαδή να έχει γίνει έλεγχος σε όλο το δέντρο και να μη βρέθηκε κόμβος που περιέχει τη λέξη \en word \gr. Στο τέλος της συνάρτησης \en search \gr (του ιδιωτικού τμήματος), επιστρέφεται ο κόμβος προς έλεγχο που χρησιμοποιήθηκε στην επανάληψη. H \en public search \gr ελέγχει αν ο κόμβος που επιστράφηκε από την \en private \gr είναι \en nullptr \gr, δηλαδή δεν βρέθηκε η λέξη \en word \gr οπότε και επιστρέφει αναφορικά μηδέν. Διαφορετικά επιστρέφεται και πάλι αναφορικά η συχνότητα εμφάνισης της λέξης.

\subsection{Μέθοδος διαγραφής}

Η διαδικασία της διαγραφής ενός κόμβου από το δέντρο αποτελεί μία περίπλοκη διαδικασία η οποία απλουστεύεται λόγω της χρήσης του κόμβου γονέα \en (parent*) \gr.

Αρχικά αναζητούμε μέσω της συνάρτησής \en search \gr που περιγράφεται παρακάτω τον κόμβο που έχει αποθηκευμένη τη λέξη \en word \gr. Όταν βρεθεί γίνονται οι απαραίτητοι έλεγχοι ώστε να διαπιστωθεί αν ο κόμβος αυτός έχει παιδιά (δηλαδή δεν είναι φύλλο), κι αν ναι πόσα ή είναι η ρίζα του δέντρου. Οι παραπάνω έλεγχοι γίνονται ώστε κατά τη διαδικασία της διαγραφής του κόμβου να γνωρίζουμε ποιος είναι ο γονέας και τα παιδιά του έτσι ώστε ο κόμβους που ενδεχομένως θα πάρει τη θέση του να μη παραβιάζει τις αρχές του δυαδικού δέντρου αναζήτησης σχετικά με τη θέση των κόμβων μέσα σε αυτό.

Στη μέθοδο διαγραφής του \en AVL \gr υπάρχει μία διαφοροποίηση καθώς η εισαγωγή κόμβου δεν υλοποιείται με τη χρήση του κόμβου γονέα \en (parent*) \gr. Έτσι όταν θέλουμε να διαγράψουμε έναν κόμβο δεν μπορούμε απλώς αναζητώντας τον να τον διαγράψουμε αφού δεν γνωρίζουμε τον γονέα του. Αυτό το δια χειριστήκαμε εκτελώντας αναδρομική διαδικασία μέσα στη συνάρτηση διαγραφής του \en AVL \gr προσπαθώντας να βρεθεί το στοιχείο, επαναλαμβάνοντας και παίρνοντας ως κόμβο προς έλεγχο το αριστερό ή δεξί παιδί της ρίζας που ελέγχεται. Όταν και αν βρεθεί σταματούν οι αναδρομικές κλήσεις και ελέγχεται πόσα παιδιά έχει ο κόμβος προς διαγραφή, αν είναι η ρίζα του δέντρου, αν είναι το παιδί κάποιας ρίζας κι αν ναι ποιο. Τότε η διαγραφή γίνεται μια εύκολη διαδικασία κι αφού ολοκληρωθεί, καλείται η συνάρτηση εξισορρόπησης ώστε όλοι οι κόμβοι (από αυτόν που διαγράφηκε έως τη ρίζα) να διατηρούν τις αρχές ισοζυγισμού του δέντρου τύπου \en AVL \gr.

\subsection{Σχολιασμός και χρόνοι εκτέλεσης}

Η διαμόρφωση του τελικού απλού δυαδικού δέντρου αναζήτησης οφείλεται στη σειρά εισαγωγής στοιχείων. Γι αυτόν το λόγο
παρατηρούμε ότι η δημιουργία του απλού δυαδικού δέντρου αναζήτησης σε σχέση με το \en AVL \gr απαιτεί λιγότερο χρόνο καθώς επίσης λόγω του ότι δεν υλοποιείται η εξισορρόπηση του δέντρου. Η εξισορρόπηση είναι εμφανώς απαραίτητη στο \en AVL \gr και απαιτεί περαιτέρω χρόνο για τους ελέγχους που αφορούν τα ύψη των υποδέντρων για κάποια ρίζα κι έπειτα την εξισορρόπηση τους μέσω της λειτουργίας των περιστροφών.

\gr
\begin{table}[!h]
\centering
\begin{tabular}{||c c c||} 
 \hline
 Δομή & Χρόνος Εισαγωγής & Χρόνος Αναζήτησης/10χιλ. λέξεις \\
 \hline\hline
  Απλό δυαδικό δένδρο & $\approx$4.4155 λεπτά & 42.901 δευτερόλεπτα \\
  Δυαδικό δένδρο τύπου \en AVL \gr & 16.0635 λεπτά & 28.382 δευτερόλεπτα \\
 \hline
\end{tabular}
\caption{Χρόνοι εκτέλεσης εισαγωγής και αναζήτησης για τις δυο δεντρικές δομές}
\label{table:15}
\end{table}

Η μέθοδος αναζήτησης είναι η ίδια και για τα δύο δέντρα, απλό δυαδικό δέντρο αναζήτησης και \en AVL \gr. Παρόλα αυτά στο απλό δυαδικό δέντρο αναζήτησης ενδέχεται το ένα υποδέντρο της ρίζας να είναι κατά πολύ μεγαλύτερο του άλλου, αφού τα στοιχεία εισάγονται με τη σειρά που λαμβάνονται από το αρχείο δίχως περιστροφές ,κι έτσι η λειτουργία της αναζήτησης απαιτεί περισσότερο χρόνο σε σχέση με αυτήν του \en AVL \gr που όλα τα υποδέντρα έχουν ύψη σε απόλυτη τιμή με διαφορά το πολύ ένα($|\mbox{Αριστερό υποδέντρο} - \mbox{δεξί υποδέντρο}| \leq  1$). Στο \en AVL \gr στην γενική περίπτωση κατά τη διάσχιση του δέντρου για την εύρεση ενός κόμβου αγνοείται μεγαλύτερο πλήθος κόμβων, λόγω του ότι είναι ισορροπημένο γλιτώνοντας χρόνο στη λειτουργία της αναζήτησης. \clearpage
\section{Πίνακας κατακερματισμού με ανοικτή διεύθυνση}
\subsection*{Περιγραφή}

Ενώ ο στόχος μιας συνάρτησης κατακερματισμού είναι η ελαχιστοποίηση των συγκρούσεων, ορισμένες συγκρούσεις είναι αναπόφευκτες στην πράξη. Έτσι, οι εφαρμογές κατακερματισμού πρέπει να περιλαμβάνουν κάποια μορφή πολιτικής επίλυσης σύγκρουσης. Οι τεχνικές επίλυσης σύγκρουσης μπορούν να χωριστούν σε δύο κατηγορίες: ανοιχτό κατακερματισμό και κλειστό κατακερματισμό (ονομάζεται επίσης ανοιχτής διεύθυνσης). Η διαφορά μεταξύ των δύο έχει να κάνει με το εάν οι συγκρούσεις αποθηκεύονται εκτός πίνακα (δηλαδή με συνδεδεμένες λίστες) ή εάν οι συγκρούσεις έχουν ως αποτέλεσμα την αποθήκευση στον ίδιο πίνακα.

Ωστόσο, το θέμα των συγκρούσεων θα μπορούσε να βελτιωθεί, αν και αυτό δεν σημαίνει ότι θα αποτρέψουμε όλες τις συγκρούσεις, αλλά μπορούμε να μειώσουμε την πιθανότητα δυο λέξεις να συγκρουστούν, δηλαδή να έχουν το ίδιο \en hash key \gr. Μια καλή και ευρέως χρησιμοποιούμενη συνάρτηση υπολογισμού του \en hash \gr μιας συμβολοσειράς βρίσκεται παρακάτω (με κάποιες βελτιώσεις - βάση του αλγορίθμου μας).

\bigskip

\noindent Έστω $s$ μια συμβολοσειρά και $q$ το μέγεθος του πίνακα. Έχουμε:

\begin{align*}
    hash(s) &= \frac{1}{q} \left( s[0] + s[1] \cdot p + s[2] \cdot p^2 + \ldots + s[n-1] \cdot p^{n-1} \mod m \right) \\
    &=  \frac{1}{q} \left( \sum_{i=0}^{n-1} s[i] \cdot p^i \mod m \right)
\end{align*}

\noindent όπου $p = 31$ και $m = 1e9 + 9$ \footnote[1]{Περισσότερα για την συνάρτησή στο άρθρο: \en https://cp-algorithms.com/string/string-hashing.html \gr }.

\bigskip

\en
\begin{listing}[ht]
\begin{minted}[frame=lines, framesep=1em]{cpp}

#pragma once
#include <string>

using std::string;

class HashNode
{
public:
    HashNode(string word, int key, int times)
    {
        this->word = word;
        this->key = key;
        this->times = times;
    }
    string word;
    unsigned int key, times;
};

\end{minted}
\caption{Υλοποίηση του \en HashNode \gr για τον πίνακα κατακερματισμού με ανοιχτή διεύθυνση}
\label{listing:9}
\end{listing}
\gr

Για την υλοποίηση της δομής η λογική που ακολουθήθηκε είναι να δημιουργήσουμε ένα  \en Hash Node \gr όπου θα αποθηκεύουμε την λέξη, το κλειδί και τις φορές που υπάρχει μέσα στην δομή. Ουσιαστικά, με τη βοήθεια του \en Hash Node \gr θα μπορούμε να διαχειριζόμαστε δυναμικά, γρηγορότερα και ευκολότερα την δομή μας.

Προχωρώντας στην υλοποίηση της κεντρικής κλάσης για τον πίνακα κατακερματισμού με ανοιχτή διεύθυνση, αρχικά, ορίζουμε ότι θέλουμε να χρησιμοποιήσουμε μια μεταβλητή που θα αποθηκεύει τον χώρο, μια τα στοιχεία τα οποία εισήχθησαν και τέλος έναν πίνακα τύπου \en HashNode \gr (βλ. Πρόγραμμα \ref{listing:9}).

Στο \en protected \gr τμήμα της κλάσης έχουμε την συνάρτηση \en getKey \gr, όπου μετατρέπει μια λέξη σε αριθμό, την \en Get \gr όπου επιστρέφει συγκεκριμένη διεύθυνση του πίνακα, την \en Hash \gr όπου υπολογίζει το κλειδί και τέλος την \en nextHash \gr όπου υπολογίζουμε το επόμενο κλειδί.

Στο \en public \gr τμήμα της κλάσης έχουμε τον κατασκευαστή, την συνάρτηση όπου θα εισάγει λέξεις μέσα στη δομή και την συνάρτηση που θα αναζητεί (βλ. Πρόγραμμα \ref{listing:10}).

\en
\begin{listing}[ht]
\begin{minted}[frame=lines, framesep=1em]{cpp}

#pragma once
#include <string>

using std::string;

class Hashing
{
private:
    unsigned long long int Size;
    unsigned long long int Elements;
    HashNode** Table;
protected:
    long long getKey(const string&);
    HashNode*& Get(const int&, const string&);
    unsigned int Hash(const int&);
    unsigned int nextHash(unsigned int);
public:
    Hashing();
    ~Hashing();
    bool Insert(const string&);
    bool Find(string&, unsigned int &);
};

\end{minted}
\caption{Υλοποίηση της κλάσης για τον πίνακα κατακερματισμού με ανοιχτή διεύθυνση}
\label{listing:10}
\end{listing}
\gr



\subsection{Μέθοδος εισαγωγής}

Η σκεπτική πλευρά της μεθόδου εισαγωγής προγραμματιστικά θα λέγαμε ότι είναι απλή (βλ. Πρόγραμμα \ref{listing:11}). Αρχικά, ελέγχουμε αν η διεύθυνση η οποία μας επέστρεψε η συνάρτηση \en Get \gr (ουσιαστικά, η διεύθυνση μιας συγκεκριμένης θέσης του πίνακα - με βάση το \en key \gr και την λέξη) είναι κενή ή όχι. Σε περίπτωση που χρησιμοποιείται τότε επιστρέφουμε \en false \gr, δηλαδή ότι δεν μπόρεσε η συγκεκριμένη λέξη να εισαχθεί. Διαφορετικά προχωρούμε κανονικά ελέγχοντας αν μας επαρκεί η μνήμη που έχουμε. Αν έχει ήδη γεμίσει το μισό του πίνακα τότε διπλασιάζουμε τον χώρο με κύριο κριτήριο οι λέξεις να εισαχθούν στον νέο πίνακα με την περισσότερη μνήμη με βάση την συνάρτησή \en Hash \gr, για να μπορούμε οποιαδήποτε στιγμή να αναζητήσουμε οποιοδήποτε στοιχείο επιθυμούμε. Στη συνέχεια εκχωρούμε την λέξη. Η περίπτωση μια λέξη να υπάρχει ήδη στον πίνακα ελέγχεται από την συνάρτηση \en Get \gr την οποία θα δούμε στο κεφάλαιο με την μέθοδο αναζήτησης.

\en
\begin{listing}[ht]
\begin{minted}[frame=lines, framesep=1em]{cpp}

bool Hashing::Insert(const string& word)
{
    int key = getKey(word);

    if (Get(key, word) != nullptr)
        return false;

    if (Elements * 2 >= Size)
    {
        Size *= 2;
        HashNode** temp;

        temp = new (std::nothrow) HashNode*[Size];
        for (int i = 0; i < Size; i++)
            temp[i] = nullptr;

        for (int i = 0; i < Size / 2; i++)
        {
            if (Table[i] != nullptr)
            {
                unsigned int p = Hash(Table[i]->key);

                while (temp[p] != nullptr)
                    p = nextHash(p);

                temp[p] = Table[i];
                Table[i] = nullptr;
            }
        }

        delete[] Table;
        Table = temp;
    }

    unsigned int HashIndex = Hash(key);
    while (Table[HashIndex] != nullptr)
        HashIndex = nextHash(HashIndex);

    Table[HashIndex] = new HashNode(word, key, 0); Elements++;
    return true;
}

\end{minted}
\caption{Δημιουργία της συνάρτησής \en Insert \gr του πίνακα κατακερματισμού με ανοιχτή διεύθυνση}
\label{listing:11}
\end{listing}
\gr


\subsection{Μέθοδος αναζήτησής}

Όπως φαίνεται και στο Πρόγραμμα \ref{listing:12}, η συνάρτηση \en Find \gr δέχεται δύο ορίσματα. Το πρώτο όρισμα δέχεται τη λέξη την οποία θέλουμε να εξετάσουμε αν υπάρχει μέσα στη δομή (η συνάρτηση δέχεται τη λέξη αναφορικά για να μη γίνει άσκοπη αντιγραφή) και το δεύτερο όρισμα τη μεταβλητή που χρησιμοποιούμε για να επιστρέψουμε το σύνολο εμφανίσεων της κάθε λέξης στο σύνολο όλων των λέξεων που έχει αποθηκεύσει η δομή. Η επιστροφή του συνόλου εμφανίσεων μιας λέξης γίνεται με την χρήση αναφορικής μεταβλητής.

Εντός της συνάρτησης, αρχικά ορίζουμε σε μια μεταβλητή το \en HashKey \gr που επιστρέφεται από την συνάρτηση \en Hash \gr και έπειτα δημιουργούμε μια μεταβλητή που θα αποθηκεύει διεύθυνση που επιστρέφει η συνάρτηση \en Get \gr. Ο τρόπος λειτουργίας της συνάρτησης \en Get \gr είναι ο εξής:

\begin{enumerate}
    \item Σε μια μεταβλητή $p$, αποθήκευσε το \en Hash Key \gr.
    \item Καθώς κάθε θέση του πίνακα έχει στοιχεία εξέτασε για κάθε θέση του πίνακα αν η λέξη που είναι αποθηκευμένη είναι ίδια με αυτή που υπάρχει ως παράμετρο. Αν ισχύει, τότε αύξησε το πλήθος της λέξης κατά μια μονάδα και επέστρεψε τη θέση του πίνακα με τις ιδιότητες του. Διαφορετικά, πήγαινε στην επόμενη θέση με βάση τη συνάρτηση \en nextHash \gr.
    \item Σε περίπτωση που δεν ισχύει το παραπάνω, τότε επιστρέφουμε τη θέση του πίνακα με τις ιδιότητες της.
\end{enumerate}

Στη συνέχεια, αφού έχει κληθεί και έχει επιστραφεί κάποιο αποτέλεσμα από την συνάρτηση \en Get \gr, εξετάζουμε αν η μεταβλητή $p$ δείνχει σε κάποια διεύθυνση θέσης του πίνακα ή όχι. Αν δεν δείχνει κάπου τότε επιστρέφουμε \en false \gr, διαφορετικά με αναφορά εκχωρούμε τις τιμές στις αντίστοιχες αναφορικές μεταβλητές \en times \gr και \en word \gr. Τέλος επιστρέφουμε \en true \gr.

\en
\begin{listing}[ht]
\begin{minted}[frame=lines, framesep=1em]{cpp}

bool Hashing::Find(string& word, unsigned int& times)
{
    int key = getKey(word);
    HashNode* p = Get(key, word);

    if (p == nullptr)
        return false;

    times = p->times;
    word = p->word;

    return true;
}

HashNode*& Hashing::Get(const int& key, const string& word)
{
    unsigned int p = Hash(key);
    while (Table[p] != nullptr)
    {
        if (Table[p]->word == word
        {
            Table[p]->times++;
            return Table[p];
        }

        p = nextHash(p);
    }

    return Table[p];
}

\end{minted}
\caption{Δημιουργία της συνάρτησής \en Find \gr και \en Get \gr του πίνακα κατακερματισμού με ανοιχτή διεύθυνση}
\label{listing:12}
\end{listing}
\gr

\subsection{Σχολιασμός και χρόνοι εκτέλεσης}

Ο χρόνος εισαγωγής σε έναν έναν πίνακα κατακερματισμού είναι λιγότερος σε σχέση με όλες τις προηγούμενες δομές. Ο καθορισμός της θέσης που θα αποθηκευτεί το κάθε στοιχείο ορίζεται πολυωνυμικά με βάση την παραπάνω συνάρτηση που υπάρχει στην περιγραφή. Σε περίπτωση, όπως αναφέρθηκε, που δεν βρεθεί ψάχνει σε άλλη -- επόμενη -- θέση, με βάση τη συνάρτηση \en nextHash \gr (βλ. Πρόγραμμα \ref{listing:12}) όπου δεν 'κοστίζει' πολύ σε πολυπλοκότητα.

\gr
\begin{table}[!h]
\centering
\begin{tabular}{||c c c||} 
 \hline
 Δομή & Χρόνος Εισαγωγής & Χρόνος Αναζήτησης/10χιλ. λέξεις \\
 \hline\hline
  Πίνακας κατακερματισμού με ανοιχτή διεύθυνση & 48.971 δευτερόλεπτα & 7.552 δευτερόλεπτα \\
 \hline
\end{tabular}
\caption{Χρόνοι εκτέλεσης εισαγωγής και αναζήτησης για τον πίνακα κατακερματισμού με ανοιχτή διεύθυνση}
\label{table:5}
\end{table}

Όσο για την αναζήτηση, όπου η διαδικασία που ακολουθείται που ακολουθείτε είναι ακριβώς η ίδια (Αλγοριθμικά περιγράφεται στο Πρόγραμμα \ref{listing:12}) ο χρόνος σε σχέση με τις προηγούμενες τέσσερις δομές έχει μεγάλη απόκλιση, λόγω της δομής του πίνακα κατακερματισμού. Μεγάλη βελτίωση στην εισαγωγή πραγματοποίησε η συνάρτηση όπου αναφέρεται στην περιγραφή, λόγω λιγότερων συγκρούσεων.  \clearpage

\section*{Τεχνικές λεπτομέρειες}

Στον Πίνακα \ref{table:6} θα βρείτε μερικές χρήσιμες πληροφορίες σχετικά με το πρότζεκτ. Όπως αναφέρεται στον παρακάτω πίνακα, το πρότζεκτ έχει δοκιμαστεί επιτυχώς στις παρακάτω τρεις πλατφόρμες-προγράμματα (\en IDE \gr και \en online IDE\gr).

\gr
\begin{table}[!h]
\centering
\begin{tabular}{||c c | c||} 
 \hline
 Γλώσσα & Έκδοση & Δοκιμάστηκε \\
 \hline\hline
  \en C++ \gr & \en C++14 \gr & \en Repl.it \gr (Μικρό αρχείο), \en Visual Studio 2019 \gr και \en *Clion 2021.1.1 \gr (Μεγάλο αρχείο) \\
 \hline
\end{tabular}
\caption{Πίνακας με βασικές τεχνικές λεπτομέρειες}
\label{table:6}
\end{table}

\noindent *Οι δομές δοκιμάστηκαν με το πρόγραμμα \en Clion 2021.1.1 \gr (για το μεγάλο αρχείο) σε λειτουργικό σύστημα \en Linux (Ubuntu 20.04 LTS) \gr και \en Windows 10 Home (Έκδοση 20H2) \gr .

\end{document}