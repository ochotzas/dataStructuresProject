\subsection*{Περιγραφή}

Ένα απλό δυαδικό δέντρο όπως και ένα δένδρο τύπου \en AVL \gr  είναι μία δενδρική δομή δεδομένων η οποία αποτελείται από κόμβους που, όπως και στα δυαδικά δέντρα, έχουν το πολύ δύο παιδιά. Αυτό που καθιστά το δυαδικό δέντρο αναζήτησης διαφορετικό από τα απλά δυαδικά δέντρα είναι το γεγονός ότι οι κόμβοι έχουν πάντα ως αριστερό παιδί αυτό με μικρότερη τιμή σε σχέση με τη ρίζα και δεξί παιδί αυτό με μεγαλύτερη τιμή σε σχέση με τη ρίζα. Το παραπάνω ισχύει -- με αναδρομική διαδικασία -- για όλα τα υποδέντρα που περιέχονται στο βασικό δυαδικό δέντρο αναζήτησης. Η διαφορά ανάμεσα σε απλό και δένδρο τύπου \en AVL \gr είναι στα ύψη των δύο θυγατρικών υποδέντρων οποιουδήποτε κόμβου, που διαφέρουν το πολύ κατά ένα.

Όπως προαναφέρθηκε τα δέντρα αποτελούνται από κόμβους, οπότε πριν προχωρήσουμε στις λειτουργίες τις εισαγωγής/αναζήτησής/διαγραφής ας δούμε τι περιέχει το \en header \gr  αρχείο (\en node.h \gr) που αφορά τους κόμβους που αποθηκεύουμε στο δέντρο (βλ. Πρόγραμμα \ref{listing:13}).

Αρχικά χρειαζόμαστε ένα \en struct \gr ώστε να αποθηκεύουμε για κάθε κόμβο τη λέξη που επιθυμούμε και τις φορές εμφάνισης της μέσα στη δομή. Έπειτα ως μέλη της κλάσης χρειαζόμαστε τρεις μεταβλητές τύπου \en node \gr που αφορούν το δεξί  και το αριστερό παιδί κάποιας ρίζας και μια \en parent \gr που αφορά τον κόμβο-ρίζα που είναι ο γονέας ενός κόμβου για κάποιο υποδέντρο. Τέλος υπάρχουν δύο κατασκευαστές: Ένας κενός που δημιουργεί κενό κόμβο και ένας δεύτερος με όρισμα ένα \en string \gr που χρησιμοποιείται στην εισαγωγή ενός νέου κόμβου. 

Όλα τα παραπάνω είναι δηλωμένα στο δημόσιο τμήμα της κλάσης παρ’ όλο που δεν χρησιμοποιούνται στη \en main \gr. Αυτό συμβαίνει για απλούστευση του κώδικα.

\en
\begin{listing}[ht]
\begin{minted}[frame=lines, framesep=1em]{cpp}

#pragma once
#include <string>

using std::string;

struct node_info
{
    string value;
    long long int times;
};

class node
{
public:
    node_info data;
    long long int height;
    node* left;
    node* right;
    node* parent;
    node(const string&);
    node();
};

\end{minted}
\caption{Υλοποίηση του struct info και της κλάσης node για τις δενδρικές δομές}
\label{listing:13}
\end{listing}
\gr


\subsection{Μέθοδος εισαγωγής}

Η εισαγωγή σε ένα απλό δυαδικό δέντρο αναζήτησης και ενός δένδρου τύπου \en AVL \gr αποτελεί μια λειτουργία/μέθοδο με πολλαπλούς ελέγχους που διευκολύνουν στη συνέχεια τη μέθοδο αναζήτησης και διαγραφής.

Στον κώδικα της μεθόδου εισαγωγής, αρχικά ελέγχεται αν το δέντρο είναι κενό οπότε και δημιουργείται η ρίζα του δέντρου που περιέχει τη λέξη \en word \gr και η μεταβλητή με τις φορές εμφάνισης τίθεται σε ένα. Για κάθε επόμενο κόμβο που θέλουμε να δημιουργηθεί, ελέγχουμε αν η λέξη (η οποία είναι το κλειδί για τους ελέγχους στην εισαγωγή), είναι μικρότερη ή μεγαλύτερη από τη ρίζα του εκάστοτε δέντρου ή υποδέντρου. Οι έλεγχοι αυτοί επαναλαμβάνονται μέχρι τη στιγμή που θα βρεθεί ένας κόμβος κατάλληλος για την δημιουργία του νέου κόμβου ως παιδί του. Αυτό προϋποθέτει να έχει ο κόμβος-γονέας αριστερό ή δεξί παιδί (ανάλογα τη θέση που πρέπει να εισαχθεί ο νέος κόμβος), ίσο με \en nullptr \gr. Σε αυτήν την περίπτωση έχουμε επιτυχημένη εισαγωγή και η συνάρτηση \en insert \gr τερματίζει επιστρέφοντας τη λογική τιμή \en true \gr. Αξιοσημείωτη είναι η περίπτωση που υπάρχει ήδη ένας κόμβος με αποθηκευμένη λέξη τη \en word \gr που θέλουμε να εισάγουμε. Τη περίπτωση αυτή τη διαχειριζόμαστε αυξάνοντας μόνο τον μετρητή που δείχνει τις φορές εμφάνισης της λέξης κατά μια μονάδα. Έτσι βεβαιωνόμαστε ότι κάθε κόμβος του δέντρου αποθηκεύει μόνο τις λέξεις που είναι άγνωστες για τη δομή. Η διαφορά μεταξύ των δυο δένδρων είναι ότι στο τέλος κάθε εισαγωγής, το δένδρο τύπου \en AVL \gr καλεί τη συνάρτησή εξισορρόπησής αναδρομικά από το κόμβο που εισήχθη έως τη ρίζα του δένδρου, ακολουθώντας το μονοπάτι που ακολουθήθηκε στην \en insert \gr για την εισαγωγή του.  

\subsection{Μέθοδος αναζήτησής}

Η διαδικασία της αναζήτησης σε ένα απλό δυαδικό δέντρο αναζήτησης όπως και σε ένα δένδρο τύπου \en AVL \gr δε διαφέρει και πολύ σε σχέση με την δυαδική αναζήτηση σε έναν ταξινομημένο πίνακα.

Ο κόμβος αναζητείται με βάση την λέξη-κλειδί που έχει αποθηκευμένη. Έτσι ξεκινώντας από τη ρίζα του δέντρου, μέσα σε μία επανάληψη ελέγχουμε αν η λέξη-κλειδί για την αναζήτηση είναι μεγαλύτερη ή μικρότερη της λέξης που έχει αποθηκευμένη η ρίζα (αρχικά στον πρώτο έλεγχο), ώστε στην επόμενη επανάληψη να λάβουμε ως κόμβο προς έλεγχο το αριστερό (ή το δεξί) παιδί της ρίζας, αγνοώντας το δεξί (ή το αριστερό υποδέντρο) αντίστοιχα. Η παραπάνω διαδικασία επαναλαμβάνεται μέχρι να βρεθεί ο κόμβος που περιέχει τη λέξη \en word \gr ή να τελειώσουν οι κόμβοι προς έλεγχο, δηλαδή να έχει γίνει έλεγχος σε όλο το δέντρο και να μη βρέθηκε κόμβος που περιέχει τη λέξη \en word \gr. Στο τέλος της συνάρτησης \en search \gr (του ιδιωτικού τμήματος), επιστρέφεται ο κόμβος προς έλεγχο που χρησιμοποιήθηκε στην επανάληψη. H \en public search \gr ελέγχει αν ο κόμβος που επιστράφηκε από την \en private \gr είναι \en nullptr \gr, δηλαδή δεν βρέθηκε η λέξη \en word \gr οπότε και επιστρέφει αναφορικά μηδέν. Διαφορετικά επιστρέφεται και πάλι αναφορικά η συχνότητα εμφάνισης της λέξης.

\subsection{Μέθοδος διαγραφής}

Η διαδικασία της διαγραφής ενός κόμβου από το δέντρο αποτελεί μία περίπλοκη διαδικασία η οποία απλουστεύεται λόγω της χρήσης του κόμβου γονέα \en (parent*) \gr.

Αρχικά αναζητούμε μέσω της συνάρτησής \en search \gr που περιγράφεται παρακάτω τον κόμβο που έχει αποθηκευμένη τη λέξη \en word \gr. Όταν βρεθεί γίνονται οι απαραίτητοι έλεγχοι ώστε να διαπιστωθεί αν ο κόμβος αυτός έχει παιδιά (δηλαδή δεν είναι φύλλο), κι αν ναι πόσα ή είναι η ρίζα του δέντρου. Οι παραπάνω έλεγχοι γίνονται ώστε κατά τη διαδικασία της διαγραφής του κόμβου να γνωρίζουμε ποιος είναι ο γονέας και τα παιδιά του έτσι ώστε ο κόμβους που ενδεχομένως θα πάρει τη θέση του να μη παραβιάζει τις αρχές του δυαδικού δέντρου αναζήτησης σχετικά με τη θέση των κόμβων μέσα σε αυτό.

Στη μέθοδο διαγραφής του \en AVL \gr υπάρχει μία διαφοροποίηση καθώς η εισαγωγή κόμβου δεν υλοποιείται με τη χρήση του κόμβου γονέα \en (parent*) \gr. Έτσι όταν θέλουμε να διαγράψουμε έναν κόμβο δεν μπορούμε απλώς αναζητώντας τον να τον διαγράψουμε αφού δεν γνωρίζουμε τον γονέα του. Αυτό το δια χειριστήκαμε εκτελώντας αναδρομική διαδικασία μέσα στη συνάρτηση διαγραφής του \en AVL \gr προσπαθώντας να βρεθεί το στοιχείο, επαναλαμβάνοντας και παίρνοντας ως κόμβο προς έλεγχο το αριστερό ή δεξί παιδί της ρίζας που ελέγχεται. Όταν και αν βρεθεί σταματούν οι αναδρομικές κλήσεις και ελέγχεται πόσα παιδιά έχει ο κόμβος προς διαγραφή, αν είναι η ρίζα του δέντρου, αν είναι το παιδί κάποιας ρίζας κι αν ναι ποιο. Τότε η διαγραφή γίνεται μια εύκολη διαδικασία κι αφού ολοκληρωθεί, καλείται η συνάρτηση εξισορρόπησης ώστε όλοι οι κόμβοι (από αυτόν που διαγράφηκε έως τη ρίζα) να διατηρούν τις αρχές ισοζυγισμού του δέντρου τύπου \en AVL \gr.

\subsection{Σχολιασμός και χρόνοι εκτέλεσης}

Η διαμόρφωση του τελικού απλού δυαδικού δέντρου αναζήτησης οφείλεται στη σειρά εισαγωγής στοιχείων. Γι αυτόν το λόγο
παρατηρούμε ότι η δημιουργία του απλού δυαδικού δέντρου αναζήτησης σε σχέση με το \en AVL \gr απαιτεί λιγότερο χρόνο καθώς επίσης λόγω του ότι δεν υλοποιείται η εξισορρόπηση του δέντρου. Η εξισορρόπηση είναι εμφανώς απαραίτητη στο \en AVL \gr και απαιτεί περαιτέρω χρόνο για τους ελέγχους που αφορούν τα ύψη των υποδέντρων για κάποια ρίζα κι έπειτα την εξισορρόπηση τους μέσω της λειτουργίας των περιστροφών.

\gr
\begin{table}[!h]
\centering
\begin{tabular}{||c c c||} 
 \hline
 Δομή & Χρόνος Εισαγωγής & Χρόνος Αναζήτησης/10χιλ. λέξεις \\
 \hline\hline
  Απλό δυαδικό δένδρο & $\approx$4.4155 λεπτά & 42.901 δευτερόλεπτα \\
  Δυαδικό δένδρο τύπου \en AVL \gr & 16.0635 λεπτά & 28.382 δευτερόλεπτα \\
 \hline
\end{tabular}
\caption{Χρόνοι εκτέλεσης εισαγωγής και αναζήτησης για τις δυο δεντρικές δομές}
\label{table:15}
\end{table}

Η μέθοδος αναζήτησης είναι η ίδια και για τα δύο δέντρα, απλό δυαδικό δέντρο αναζήτησης και \en AVL \gr. Παρόλα αυτά στο απλό δυαδικό δέντρο αναζήτησης ενδέχεται το ένα υποδέντρο της ρίζας να είναι κατά πολύ μεγαλύτερο του άλλου, αφού τα στοιχεία εισάγονται με τη σειρά που λαμβάνονται από το αρχείο δίχως περιστροφές ,κι έτσι η λειτουργία της αναζήτησης απαιτεί περισσότερο χρόνο σε σχέση με αυτήν του \en AVL \gr που όλα τα υποδέντρα έχουν ύψη σε απόλυτη τιμή με διαφορά το πολύ ένα($|\mbox{Αριστερό υποδέντρο} - \mbox{δεξί υποδέντρο}| \leq  1$). Στο \en AVL \gr στην γενική περίπτωση κατά τη διάσχιση του δέντρου για την εύρεση ενός κόμβου αγνοείται μεγαλύτερο πλήθος κόμβων, λόγω του ότι είναι ισορροπημένο γλιτώνοντας χρόνο στη λειτουργία της αναζήτησης.